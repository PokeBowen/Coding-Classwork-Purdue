% Options for packages loaded elsewhere
\PassOptionsToPackage{unicode}{hyperref}
\PassOptionsToPackage{hyphens}{url}
\documentclass[
]{article}
\usepackage{xcolor}
\usepackage[margin=1in]{geometry}
\usepackage{amsmath,amssymb}
\setcounter{secnumdepth}{-\maxdimen} % remove section numbering
\usepackage{iftex}
\ifPDFTeX
  \usepackage[T1]{fontenc}
  \usepackage[utf8]{inputenc}
  \usepackage{textcomp} % provide euro and other symbols
\else % if luatex or xetex
  \usepackage{unicode-math} % this also loads fontspec
  \defaultfontfeatures{Scale=MatchLowercase}
  \defaultfontfeatures[\rmfamily]{Ligatures=TeX,Scale=1}
\fi
\usepackage{lmodern}
\ifPDFTeX\else
  % xetex/luatex font selection
\fi
% Use upquote if available, for straight quotes in verbatim environments
\IfFileExists{upquote.sty}{\usepackage{upquote}}{}
\IfFileExists{microtype.sty}{% use microtype if available
  \usepackage[]{microtype}
  \UseMicrotypeSet[protrusion]{basicmath} % disable protrusion for tt fonts
}{}
\makeatletter
\@ifundefined{KOMAClassName}{% if non-KOMA class
  \IfFileExists{parskip.sty}{%
    \usepackage{parskip}
  }{% else
    \setlength{\parindent}{0pt}
    \setlength{\parskip}{6pt plus 2pt minus 1pt}}
}{% if KOMA class
  \KOMAoptions{parskip=half}}
\makeatother
\usepackage{color}
\usepackage{fancyvrb}
\newcommand{\VerbBar}{|}
\newcommand{\VERB}{\Verb[commandchars=\\\{\}]}
\DefineVerbatimEnvironment{Highlighting}{Verbatim}{commandchars=\\\{\}}
% Add ',fontsize=\small' for more characters per line
\usepackage{framed}
\definecolor{shadecolor}{RGB}{248,248,248}
\newenvironment{Shaded}{\begin{snugshade}}{\end{snugshade}}
\newcommand{\AlertTok}[1]{\textcolor[rgb]{0.94,0.16,0.16}{#1}}
\newcommand{\AnnotationTok}[1]{\textcolor[rgb]{0.56,0.35,0.01}{\textbf{\textit{#1}}}}
\newcommand{\AttributeTok}[1]{\textcolor[rgb]{0.13,0.29,0.53}{#1}}
\newcommand{\BaseNTok}[1]{\textcolor[rgb]{0.00,0.00,0.81}{#1}}
\newcommand{\BuiltInTok}[1]{#1}
\newcommand{\CharTok}[1]{\textcolor[rgb]{0.31,0.60,0.02}{#1}}
\newcommand{\CommentTok}[1]{\textcolor[rgb]{0.56,0.35,0.01}{\textit{#1}}}
\newcommand{\CommentVarTok}[1]{\textcolor[rgb]{0.56,0.35,0.01}{\textbf{\textit{#1}}}}
\newcommand{\ConstantTok}[1]{\textcolor[rgb]{0.56,0.35,0.01}{#1}}
\newcommand{\ControlFlowTok}[1]{\textcolor[rgb]{0.13,0.29,0.53}{\textbf{#1}}}
\newcommand{\DataTypeTok}[1]{\textcolor[rgb]{0.13,0.29,0.53}{#1}}
\newcommand{\DecValTok}[1]{\textcolor[rgb]{0.00,0.00,0.81}{#1}}
\newcommand{\DocumentationTok}[1]{\textcolor[rgb]{0.56,0.35,0.01}{\textbf{\textit{#1}}}}
\newcommand{\ErrorTok}[1]{\textcolor[rgb]{0.64,0.00,0.00}{\textbf{#1}}}
\newcommand{\ExtensionTok}[1]{#1}
\newcommand{\FloatTok}[1]{\textcolor[rgb]{0.00,0.00,0.81}{#1}}
\newcommand{\FunctionTok}[1]{\textcolor[rgb]{0.13,0.29,0.53}{\textbf{#1}}}
\newcommand{\ImportTok}[1]{#1}
\newcommand{\InformationTok}[1]{\textcolor[rgb]{0.56,0.35,0.01}{\textbf{\textit{#1}}}}
\newcommand{\KeywordTok}[1]{\textcolor[rgb]{0.13,0.29,0.53}{\textbf{#1}}}
\newcommand{\NormalTok}[1]{#1}
\newcommand{\OperatorTok}[1]{\textcolor[rgb]{0.81,0.36,0.00}{\textbf{#1}}}
\newcommand{\OtherTok}[1]{\textcolor[rgb]{0.56,0.35,0.01}{#1}}
\newcommand{\PreprocessorTok}[1]{\textcolor[rgb]{0.56,0.35,0.01}{\textit{#1}}}
\newcommand{\RegionMarkerTok}[1]{#1}
\newcommand{\SpecialCharTok}[1]{\textcolor[rgb]{0.81,0.36,0.00}{\textbf{#1}}}
\newcommand{\SpecialStringTok}[1]{\textcolor[rgb]{0.31,0.60,0.02}{#1}}
\newcommand{\StringTok}[1]{\textcolor[rgb]{0.31,0.60,0.02}{#1}}
\newcommand{\VariableTok}[1]{\textcolor[rgb]{0.00,0.00,0.00}{#1}}
\newcommand{\VerbatimStringTok}[1]{\textcolor[rgb]{0.31,0.60,0.02}{#1}}
\newcommand{\WarningTok}[1]{\textcolor[rgb]{0.56,0.35,0.01}{\textbf{\textit{#1}}}}
\usepackage{graphicx}
\makeatletter
\newsavebox\pandoc@box
\newcommand*\pandocbounded[1]{% scales image to fit in text height/width
  \sbox\pandoc@box{#1}%
  \Gscale@div\@tempa{\textheight}{\dimexpr\ht\pandoc@box+\dp\pandoc@box\relax}%
  \Gscale@div\@tempb{\linewidth}{\wd\pandoc@box}%
  \ifdim\@tempb\p@<\@tempa\p@\let\@tempa\@tempb\fi% select the smaller of both
  \ifdim\@tempa\p@<\p@\scalebox{\@tempa}{\usebox\pandoc@box}%
  \else\usebox{\pandoc@box}%
  \fi%
}
% Set default figure placement to htbp
\def\fps@figure{htbp}
\makeatother
\setlength{\emergencystretch}{3em} % prevent overfull lines
\providecommand{\tightlist}{%
  \setlength{\itemsep}{0pt}\setlength{\parskip}{0pt}}
\usepackage{bookmark}
\IfFileExists{xurl.sty}{\usepackage{xurl}}{} % add URL line breaks if available
\urlstyle{same}
\hypersetup{
  pdftitle={STAT 526 HW 1},
  hidelinks,
  pdfcreator={LaTeX via pandoc}}

\title{STAT 526 HW 1}
\author{}
\date{\vspace{-2.5em}}

\begin{document}
\maketitle

\section{STAT 526 HW 1}\label{stat-526-hw-1}

\subsection{Problem 0}\label{problem-0}

Name

Bowen Zheng

\subsection{Problem 1}\label{problem-1}

A multiple regression, involving 88 cases and 6 predictors, resulted in
an \(R^2\) = 0.48. Based on this information, what is the F statistic,
its degrees of freedom, and P-value?

We have 6 predictors with n = 88. The degree of freedom is \(df_1 = 6\).
The degree of freedom error is \(df_2 = 88 - (6+1) = 81\).

We can calculate our F-statistics with R squared.

\[ 
F = \dfrac{R^2 / df_1}{(1-R^2)/df_2} = \dfrac{0.48/6)}{(1-0.48)/81} = 12.46
\]

Our F-statistic is 12.46 with 6 and 81 degrees of freedom. We can use a
calculate and find that the corresponding P-value is approximately 0.

\subsection{Problem 2}\label{problem-2}

In Slide 18 of Topic 1, an F test is described to compare a more
flexible model with a reduced one (e.g., some parameters in the reduced
model are set to 0). When a multiple regression has replications at
certain sets of X, a lack of fit test can be performed using this same
framework. The full model does not put any assumptions on the means for
each set of X. The reduced model assumes the means are a linear function
of X. Assuming that there are C sets of X in the data set and Set i has
\(n_i\) observations (\(n_i\) \textgreater{} 1 for some sets), write out
the lack-of-fit test in this framework by specifying the full and
reduced models, their degrees of freedom, and the associated error
degrees of freedom. This total number of observations is n =
\(\sum_{i=1}^C n_i\).

\subsubsection{The full model:}\label{the-full-model}

Each set of C has their own mean. We have C sets of X.

Our equation is \$ Y\_\{ij\} = \mu\emph{i + \epsilon}\{ij\}\$ where
\(i \in 1,2,...,C\).

This means we have C number of parameters and our degree of freedom is C
and our degree freedom of error is \(n-C\).

\subsubsection{The reduced model:}\label{the-reduced-model}

We now have a linear relation / model for the means:
\(Y_{ij} = \beta_0 + \beta_1 x_{ij1} + \cdots + \beta_k x_{ijk} + \epsilon_{ij}\).

With \(k\) features from the data, we have \(k+1\) parameters used in
our model. The degree of freedom is \(k+1\) with a degree of freedom
error of \(n-k-1\).

\subsubsection{When performing the test}\label{when-performing-the-test}

Using the two models we can compute things like SSE and SSM and
ultimately, the difference in their degrees of freedom is
\((n-k-1)-(n-C) = C-k-1\), which will be used when performing the lack
of fit test.

\subsection{Problem 3}\label{problem-3}

The analysis of the Georgia data set in Topic 1 did not account for the
fact that the response variable (proportion undercount) may have
nonconstant variances. Perform a weighted regression fit of the final
model on Slide 42 assuming the binomial setting for the proportion
undercount and summarize the results. Also compare these results with
those shown in Topic 1.

We first grab the code from topic 1 so we can have access to the final
model on slide 42.

\begin{Shaded}
\begin{Highlighting}[]
\FunctionTok{library}\NormalTok{(faraway)}
\FunctionTok{data}\NormalTok{(gavote)}

\DocumentationTok{\#\#\# Lists the first 6 datalines of the data set}
\FunctionTok{head}\NormalTok{(gavote)}

\DocumentationTok{\#\#\# Summarizes the structure of the data set}
\FunctionTok{str}\NormalTok{(gavote)}

\DocumentationTok{\#\#\# Get summary statistics for each of the variables}
\FunctionTok{summary}\NormalTok{(gavote)}

\DocumentationTok{\#\#\# Because number of votes highly skewed, will look at percent undercount}
\NormalTok{percunder }\OtherTok{\textless{}{-}}\NormalTok{ (gavote}\SpecialCharTok{$}\NormalTok{ballots }\SpecialCharTok{{-}}\NormalTok{ gavote}\SpecialCharTok{$}\NormalTok{votes)}\SpecialCharTok{/}\NormalTok{gavote}\SpecialCharTok{$}\NormalTok{ballots}

\DocumentationTok{\#\#\# Generate histogram of percent undercount}
\FunctionTok{hist}\NormalTok{(percunder,}\AttributeTok{xlab=}\StringTok{"Percent"}\NormalTok{,}\AttributeTok{las=}\DecValTok{1}\NormalTok{,}\AttributeTok{main=}\StringTok{"Undercount"}\NormalTok{)}
\end{Highlighting}
\end{Shaded}

\begin{Shaded}
\begin{Highlighting}[]
\DocumentationTok{\#\#\# Generate density with data shown at bottom}
\FunctionTok{plot}\NormalTok{(}\FunctionTok{density}\NormalTok{(percunder),}\AttributeTok{main=}\StringTok{"Percent Undercount"}\NormalTok{,}\AttributeTok{las=}\DecValTok{1}\NormalTok{)}
\FunctionTok{rug}\NormalTok{(percunder)}
\end{Highlighting}
\end{Shaded}

\begin{Shaded}
\begin{Highlighting}[]
\DocumentationTok{\#\#\# Define new percent variables for Gore and Bush votes}
\NormalTok{pergore }\OtherTok{=}\NormalTok{ gavote}\SpecialCharTok{$}\NormalTok{gore}\SpecialCharTok{/}\NormalTok{gavote}\SpecialCharTok{$}\NormalTok{votes}
\NormalTok{perbush }\OtherTok{=}\NormalTok{ gavote}\SpecialCharTok{$}\NormalTok{bush}\SpecialCharTok{/}\NormalTok{gavote}\SpecialCharTok{$}\NormalTok{votes}

\DocumentationTok{\#\#\# Generate a scatterplot matrix of numeric variables}
\FunctionTok{pairs}\NormalTok{(}\SpecialCharTok{\textasciitilde{}}\NormalTok{percunder}\SpecialCharTok{+}\NormalTok{gavote}\SpecialCharTok{$}\NormalTok{perAA}\SpecialCharTok{+}\NormalTok{pergore}\SpecialCharTok{+}\NormalTok{perbush,}\AttributeTok{pch=}\DecValTok{20}\NormalTok{)}
\end{Highlighting}
\end{Shaded}

\begin{Shaded}
\begin{Highlighting}[]
\DocumentationTok{\#\#\# Generate side{-}by{-}side boxplots }
\FunctionTok{plot}\NormalTok{(percunder}\SpecialCharTok{\textasciitilde{}}\NormalTok{rural,gavote,}\AttributeTok{las=}\DecValTok{1}\NormalTok{,}\AttributeTok{ylab=}\StringTok{"Percent"}\NormalTok{)}
\end{Highlighting}
\end{Shaded}

\begin{Shaded}
\begin{Highlighting}[]
\FunctionTok{plot}\NormalTok{(percunder}\SpecialCharTok{\textasciitilde{}}\NormalTok{equip,gavote,}\AttributeTok{las=}\DecValTok{1}\NormalTok{)}
\end{Highlighting}
\end{Shaded}

\begin{Shaded}
\begin{Highlighting}[]
\DocumentationTok{\#\#\#Fitting a linear model using the lm function }
\NormalTok{model1 }\OtherTok{=} \FunctionTok{lm}\NormalTok{(percunder }\SpecialCharTok{\textasciitilde{}}\NormalTok{ pergore }\SpecialCharTok{+}\NormalTok{ perAA, gavote)}

\DocumentationTok{\#\#\#Obtain summary information from model fit}
\FunctionTok{summary}\NormalTok{(model1)}

\DocumentationTok{\#\#\#Reduced summary information function proposed by Faraway}
\FunctionTok{sumary}\NormalTok{(model1)}

\DocumentationTok{\#\#\#Requesting ANOVA Table but be wary this is using Type I SS}
\FunctionTok{anova}\NormalTok{(model1)}

\DocumentationTok{\#\#\# This library contains function allowing Type III SS.  Be wary}
\DocumentationTok{\#\#\# of its use too.  Often need to change options using}
\FunctionTok{options}\NormalTok{(}\AttributeTok{contrasts =} \FunctionTok{c}\NormalTok{(}\StringTok{"contr.sum"}\NormalTok{, }\StringTok{"contr.poly"}\NormalTok{))}

\FunctionTok{library}\NormalTok{(car)}
\FunctionTok{Anova}\NormalTok{(model1, }\AttributeTok{type=}\DecValTok{3}\NormalTok{)}

\DocumentationTok{\#\#\# Generate a 2x2 panel of diagnostic plots}
\FunctionTok{par}\NormalTok{(}\AttributeTok{mar=}\FunctionTok{c}\NormalTok{(}\DecValTok{2}\NormalTok{,}\DecValTok{2}\NormalTok{,}\DecValTok{2}\NormalTok{,}\DecValTok{2}\NormalTok{),}\AttributeTok{mfrow=}\FunctionTok{c}\NormalTok{(}\DecValTok{2}\NormalTok{,}\DecValTok{2}\NormalTok{))}
\FunctionTok{plot}\NormalTok{(model1,}\AttributeTok{cex=}\FloatTok{0.65}\NormalTok{,}\AttributeTok{cex.axis=}\FloatTok{0.7}\NormalTok{,}\AttributeTok{cex.lab=}\FloatTok{0.5}\NormalTok{)}
\end{Highlighting}
\end{Shaded}

\begin{Shaded}
\begin{Highlighting}[]
\DocumentationTok{\#\#\# Creating centered variables.  Can help with multicolinearity when}
\DocumentationTok{\#\#\# considering polynomials and interactions}
\NormalTok{cpergore }\OtherTok{=}\NormalTok{ pergore }\SpecialCharTok{{-}} \FunctionTok{mean}\NormalTok{(pergore)}
\NormalTok{cperAA }\OtherTok{=}\NormalTok{ gavote}\SpecialCharTok{$}\NormalTok{perAA }\SpecialCharTok{{-}} \FunctionTok{mean}\NormalTok{(gavote}\SpecialCharTok{$}\NormalTok{perAA)}

\DocumentationTok{\#\#\# Fitting alternative model}
\NormalTok{model2 }\OtherTok{=} \FunctionTok{lm}\NormalTok{(percunder }\SpecialCharTok{\textasciitilde{}}\NormalTok{ cperAA}\SpecialCharTok{+}\NormalTok{cpergore}\SpecialCharTok{*}\NormalTok{rural}\SpecialCharTok{+}\NormalTok{equip, gavote)}
\FunctionTok{sumary}\NormalTok{(model2)}

\DocumentationTok{\#\#\# General linear test comparing the two models}
\FunctionTok{anova}\NormalTok{(model1,model2)}

\DocumentationTok{\#\#\# Consider dropping single predictors.  Again performing general linear test}
\FunctionTok{drop1}\NormalTok{(model2,}\AttributeTok{test=}\StringTok{"F"}\NormalTok{)}

\DocumentationTok{\#\#\# New model dropping insignificant terms from previous function}
\NormalTok{model3 }\OtherTok{=} \FunctionTok{lm}\NormalTok{(percunder }\SpecialCharTok{\textasciitilde{}}\NormalTok{ cpergore}\SpecialCharTok{+}\NormalTok{rural}\SpecialCharTok{+}\NormalTok{equip, gavote)}
\FunctionTok{sumary}\NormalTok{(model3)}
\FunctionTok{anova}\NormalTok{(model2,model3)}

\DocumentationTok{\#\#\# Defining maximum model from which to select from }
\NormalTok{modelmax }\OtherTok{=} \FunctionTok{lm}\NormalTok{(percunder }\SpecialCharTok{\textasciitilde{}}\NormalTok{ (equip}\SpecialCharTok{+}\NormalTok{econ}\SpecialCharTok{+}\NormalTok{rural}\SpecialCharTok{+}\NormalTok{atlanta)}\SpecialCharTok{\^{}}\DecValTok{2} \SpecialCharTok{+}\NormalTok{ (equip}\SpecialCharTok{+}\NormalTok{econ}\SpecialCharTok{+}\NormalTok{rural}\SpecialCharTok{+}\NormalTok{atlanta)}\SpecialCharTok{*}\NormalTok{(pergore}\SpecialCharTok{+}\NormalTok{perAA), gavote)}

\DocumentationTok{\#\#\# Using AIC to reduce model}
\NormalTok{modelbest }\OtherTok{=} \FunctionTok{step}\NormalTok{(modelmax,}\AttributeTok{trace=}\ConstantTok{FALSE}\NormalTok{)}
\FunctionTok{summary}\NormalTok{(modelbest)}

\FunctionTok{drop1}\NormalTok{(modelbest,}\AttributeTok{test=}\StringTok{"F"}\NormalTok{)}

\DocumentationTok{\#\#\# Determing best model}

\NormalTok{modelbetter2 }\OtherTok{=} \FunctionTok{lm}\NormalTok{(percunder }\SpecialCharTok{\textasciitilde{}}\NormalTok{ equip}\SpecialCharTok{+}\NormalTok{econ}\SpecialCharTok{+}\NormalTok{rural}\SpecialCharTok{+}\NormalTok{perAA}\SpecialCharTok{+}\NormalTok{equip}\SpecialCharTok{:}\NormalTok{econ}\SpecialCharTok{+}\NormalTok{equip}\SpecialCharTok{:}\NormalTok{perAA, gavote)}
\FunctionTok{drop1}\NormalTok{(modelbetter2,}\AttributeTok{test=}\StringTok{"F"}\NormalTok{)}
\FunctionTok{sumary}\NormalTok{(modelbetter2)}

\DocumentationTok{\#\#\# Getting tables of predictions}
\NormalTok{pdf }\OtherTok{\textless{}{-}} \FunctionTok{data.frame}\NormalTok{(}\AttributeTok{econ=}\FunctionTok{rep}\NormalTok{(}\FunctionTok{levels}\NormalTok{(gavote}\SpecialCharTok{$}\NormalTok{econ),}\DecValTok{5}\NormalTok{),}\AttributeTok{equip=}\FunctionTok{rep}\NormalTok{(}\FunctionTok{levels}\NormalTok{(gavote}\SpecialCharTok{$}\NormalTok{equip), }\FunctionTok{rep}\NormalTok{(}\DecValTok{3}\NormalTok{,}\DecValTok{5}\NormalTok{)), }\AttributeTok{perAA=}\FloatTok{0.233}\NormalTok{, }\AttributeTok{rural=}\StringTok{"rural"}\NormalTok{)}
\NormalTok{ppr }\OtherTok{=} \FunctionTok{predict}\NormalTok{(modelbetter2,}\AttributeTok{new=}\NormalTok{pdf)}
\end{Highlighting}
\end{Shaded}

\begin{verbatim}
## Warning in predict.lm(modelbetter2, new = pdf): prediction from rank-deficient
## fit; attr(*, "non-estim") has doubtful cases
\end{verbatim}

\begin{Shaded}
\begin{Highlighting}[]
\FunctionTok{xtabs}\NormalTok{(}\FunctionTok{round}\NormalTok{(ppr,}\DecValTok{3}\NormalTok{)}\SpecialCharTok{\textasciitilde{}}\NormalTok{econ}\SpecialCharTok{+}\NormalTok{equip,pdf)}

\NormalTok{pdf }\OtherTok{\textless{}{-}} \FunctionTok{data.frame}\NormalTok{(}\AttributeTok{econ=}\FunctionTok{rep}\NormalTok{(}\FunctionTok{levels}\NormalTok{(gavote}\SpecialCharTok{$}\NormalTok{econ),}\DecValTok{5}\NormalTok{),}\AttributeTok{equip=}\FunctionTok{rep}\NormalTok{(}\FunctionTok{levels}\NormalTok{(gavote}\SpecialCharTok{$}\NormalTok{equip), }\FunctionTok{rep}\NormalTok{(}\DecValTok{3}\NormalTok{,}\DecValTok{5}\NormalTok{)), }\AttributeTok{perAA=}\FloatTok{0.233}\NormalTok{, }\AttributeTok{rural=}\StringTok{"urban"}\NormalTok{)}
\NormalTok{ppu }\OtherTok{=} \FunctionTok{predict}\NormalTok{(modelbetter2,}\AttributeTok{new=}\NormalTok{pdf)}
\end{Highlighting}
\end{Shaded}

\begin{verbatim}
## Warning in predict.lm(modelbetter2, new = pdf): prediction from rank-deficient
## fit; attr(*, "non-estim") has doubtful cases
\end{verbatim}

\begin{Shaded}
\begin{Highlighting}[]
\FunctionTok{xtabs}\NormalTok{(}\FunctionTok{round}\NormalTok{(ppu,}\DecValTok{3}\NormalTok{)}\SpecialCharTok{\textasciitilde{}}\NormalTok{econ}\SpecialCharTok{+}\NormalTok{equip,pdf)}
\end{Highlighting}
\end{Shaded}

OK now let us update the model with the weights being the count of the
ballots. If we are assuming binomial with variance
\(Var(p) = \dfrac{p(1-p)}{n}\) and use the
\(w \propto \dfrac{1}{\sigma^2}\) then the weight of the ballots should
be inversely proportional to the variance. Given that we do not know the
true proportion of undercounting, the weight then is simply just n, the
number of ballots.

\begin{Shaded}
\begin{Highlighting}[]
\NormalTok{model\_wls }\OtherTok{\textless{}{-}} \FunctionTok{update}\NormalTok{(modelbetter2, }\AttributeTok{weights =}\NormalTok{ ballots, }\AttributeTok{data =}\NormalTok{ gavote)}
\FunctionTok{summary}\NormalTok{(model\_wls)}
\end{Highlighting}
\end{Shaded}

\begin{verbatim}
## 
## Call:
## lm(formula = percunder ~ equip + econ + rural + perAA + equip:econ + 
##     equip:perAA, data = gavote, weights = ballots)
## 
## Weighted Residuals:
##     Min      1Q  Median      3Q     Max 
## -3.8195 -0.8363 -0.0707  0.8911  7.3157 
## 
## Coefficients: (2 not defined because of singularities)
##               Estimate Std. Error t value Pr(>|t|)    
## (Intercept)   0.035228   0.014556   2.420  0.01680 *  
## equip1        0.006031   0.015292   0.394  0.69389    
## equip2       -0.005891   0.015045  -0.392  0.69597    
## equip3        0.020433   0.015975   1.279  0.20301    
## equip4       -0.056013   0.057479  -0.974  0.33149    
## econ1         0.002216   0.002349   0.943  0.34705    
## econ2         0.020012   0.003753   5.332 3.80e-07 ***
## rural1        0.004483   0.001780   2.519  0.01290 *  
## perAA        -0.004733   0.040301  -0.117  0.90668    
## equip1:econ1 -0.002237   0.004207  -0.532  0.59568    
## equip2:econ1 -0.001931   0.003324  -0.581  0.56215    
## equip3:econ1 -0.009352   0.003480  -2.687  0.00808 ** 
## equip4:econ1        NA         NA      NA       NA    
## equip1:econ2 -0.004732   0.005244  -0.902  0.36838    
## equip2:econ2 -0.011625   0.004983  -2.333  0.02108 *  
## equip3:econ2  0.024526   0.005924   4.140 5.97e-05 ***
## equip4:econ2        NA         NA      NA       NA    
## equip1:perAA -0.038067   0.043786  -0.869  0.38612    
## equip2:perAA  0.055830   0.043329   1.289  0.19969    
## equip3:perAA -0.041768   0.045311  -0.922  0.35822    
## equip4:perAA  0.091829   0.156455   0.587  0.55819    
## ---
## Signif. codes:  0 '***' 0.001 '**' 0.01 '*' 0.05 '.' 0.1 ' ' 1
## 
## Residual standard error: 1.765 on 140 degrees of freedom
## Multiple R-squared:  0.6388, Adjusted R-squared:  0.5924 
## F-statistic: 13.76 on 18 and 140 DF,  p-value: < 2.2e-16
\end{verbatim}

\begin{Shaded}
\begin{Highlighting}[]
\FunctionTok{par}\NormalTok{(}\AttributeTok{mfrow=}\FunctionTok{c}\NormalTok{(}\DecValTok{1}\NormalTok{,}\DecValTok{2}\NormalTok{))}
\FunctionTok{plot}\NormalTok{(modelbetter2, }\AttributeTok{which =} \DecValTok{3}\NormalTok{, }\AttributeTok{main =} \StringTok{"OLS (Heteroscedasticity)"}\NormalTok{)}
\end{Highlighting}
\end{Shaded}

\begin{verbatim}
## Warning: not plotting observations with leverage one:
##   103, 131
\end{verbatim}

\begin{Shaded}
\begin{Highlighting}[]
\FunctionTok{plot}\NormalTok{(model\_wls, }\AttributeTok{which =} \DecValTok{3}\NormalTok{, }\AttributeTok{main =} \StringTok{"WLS (Standardized)"}\NormalTok{)}
\end{Highlighting}
\end{Shaded}

\begin{verbatim}
## Warning: not plotting observations with leverage one:
##   103, 131
\end{verbatim}

\pandocbounded{\includegraphics[keepaspectratio]{STAT526_HW1_files/figure-latex/unnamed-chunk-2-1.pdf}}

\begin{Shaded}
\begin{Highlighting}[]
\DocumentationTok{\#\#\# Getting tables of predictions}
\NormalTok{pdf }\OtherTok{\textless{}{-}} \FunctionTok{data.frame}\NormalTok{(}\AttributeTok{econ=}\FunctionTok{rep}\NormalTok{(}\FunctionTok{levels}\NormalTok{(gavote}\SpecialCharTok{$}\NormalTok{econ),}\DecValTok{5}\NormalTok{),}\AttributeTok{equip=}\FunctionTok{rep}\NormalTok{(}\FunctionTok{levels}\NormalTok{(gavote}\SpecialCharTok{$}\NormalTok{equip), }\FunctionTok{rep}\NormalTok{(}\DecValTok{3}\NormalTok{,}\DecValTok{5}\NormalTok{)), }\AttributeTok{perAA=}\FloatTok{0.233}\NormalTok{, }\AttributeTok{rural=}\StringTok{"rural"}\NormalTok{)}
\NormalTok{ppr }\OtherTok{=} \FunctionTok{predict}\NormalTok{(model\_wls,}\AttributeTok{new=}\NormalTok{pdf)}
\end{Highlighting}
\end{Shaded}

\begin{verbatim}
## Warning in predict.lm(model_wls, new = pdf): prediction from rank-deficient
## fit; attr(*, "non-estim") has doubtful cases
\end{verbatim}

\begin{Shaded}
\begin{Highlighting}[]
\FunctionTok{xtabs}\NormalTok{(}\FunctionTok{round}\NormalTok{(ppr,}\DecValTok{3}\NormalTok{)}\SpecialCharTok{\textasciitilde{}}\NormalTok{econ}\SpecialCharTok{+}\NormalTok{equip,pdf)}
\end{Highlighting}
\end{Shaded}

\begin{verbatim}
##         equip
## econ      LEVER  OS-CC  OS-PC  PAPER  PUNCH
##   middle  0.036  0.046  0.042  0.006  0.074
##   poor    0.051  0.054  0.094  0.024  0.070
##   rich    0.021  0.037  0.012 -0.018  0.031
\end{verbatim}

\begin{Shaded}
\begin{Highlighting}[]
\NormalTok{pdf }\OtherTok{\textless{}{-}} \FunctionTok{data.frame}\NormalTok{(}\AttributeTok{econ=}\FunctionTok{rep}\NormalTok{(}\FunctionTok{levels}\NormalTok{(gavote}\SpecialCharTok{$}\NormalTok{econ),}\DecValTok{5}\NormalTok{),}\AttributeTok{equip=}\FunctionTok{rep}\NormalTok{(}\FunctionTok{levels}\NormalTok{(gavote}\SpecialCharTok{$}\NormalTok{equip), }\FunctionTok{rep}\NormalTok{(}\DecValTok{3}\NormalTok{,}\DecValTok{5}\NormalTok{)), }\AttributeTok{perAA=}\FloatTok{0.233}\NormalTok{, }\AttributeTok{rural=}\StringTok{"urban"}\NormalTok{)}
\NormalTok{ppu }\OtherTok{=} \FunctionTok{predict}\NormalTok{(model\_wls,}\AttributeTok{new=}\NormalTok{pdf)}
\end{Highlighting}
\end{Shaded}

\begin{verbatim}
## Warning in predict.lm(model_wls, new = pdf): prediction from rank-deficient
## fit; attr(*, "non-estim") has doubtful cases
\end{verbatim}

\begin{Shaded}
\begin{Highlighting}[]
\FunctionTok{xtabs}\NormalTok{(}\FunctionTok{round}\NormalTok{(ppu,}\DecValTok{3}\NormalTok{)}\SpecialCharTok{\textasciitilde{}}\NormalTok{econ}\SpecialCharTok{+}\NormalTok{equip,pdf)}
\end{Highlighting}
\end{Shaded}

\begin{verbatim}
##         equip
## econ      LEVER  OS-CC  OS-PC  PAPER  PUNCH
##   middle  0.027  0.037  0.033 -0.003  0.065
##   poor    0.042  0.045  0.085  0.015  0.061
##   rich    0.012  0.028  0.003 -0.027  0.022
\end{verbatim}

We can finally view our results. They are indeed quite similar to what
we saw in topic one. The predicted percent under count is nearly
identical (at most off by less than one percent) to our OLS model.
However, it seems that the variables in our model have changed in
significance slightly. In fact, we seem to have lost some variables like
equip4:econ2.

\subsection{Problem 4}\label{problem-4}

On Slide 24 of Topic 1, you are provided a formula for the β1
coefficient given X2 is already in the model that is a function of the
β1 estimate given it is the only predictor in the model. Derive this
equation using the fact that β1 for the model Y \textbar X1, X2 can be
obtained by regressing the residuals of Y \textbar X2 vs X1\textbar X2
and the relationship between Pearson correlation and the slope in simple
linear regression.

We start by standardizing our variables. Let
\(Z_Y = \dfrac{Y-\mu_Y}{S_Y}\),
\(Z_1 = \dfrac{X_1-\mu_{X_1}}{S_{X_1}}\), and
\(Z_2 = \dfrac{X_2-\mu_{X_2}}{S_{X_2}}\). Then our residual of Y
adjusted for \(X_2\) is \(e_{y|2} = Z_Y - r_{y2}Z_2\) and the residual
of \(X_1\) adjusted for \(X_2\) is \(e_{1|2} = Z_1 - r_{12}Z_2\).

The multiple regression coefficient \(\beta_1\) is found by regressing
\(e_{y|2}\) on \(e_{1|2}\). Since the means of the residuals are 0, the
formula is \(\beta_1 = \dfrac{Cov(e_{y|2}, e_{1|2})}{Var(e_{1|2})}\).

We first solve for the denominator. We find

\begin{aligned}
\text{Var}(e_{1|2}) &= E[(Z_1 - r_{12}Z_2)^2] \\
&= E[Z_1^2 - 2r_{12}Z_1 Z_2 + r_{12}^2 Z_2^2] \\
&= E[Z_1^2] - 2r_{12}E[Z_1 Z_2] + r_{12}^2 E[Z_2^2] \\
&= 1 - 2r_{12}^2 + r_{12}^2 \\
&= 1 - r_{12}^2
\end{aligned}

Note that since the variables are standardized, we know that
\(E[Z_i^2] = 1\) and that \(E[Z_i Z_j] = r_{ij}\). Also that
\(r_{ii} = 1\).

Next we solve for the numerator.

\begin{aligned}
\text{Cov}(e_{y|2}, e_{1|2}) &= E[(Z_Y - r_{y2}Z_2)(Z_1 - r_{12}Z_2)] \\
&= E[Z_Y Z_1 - r_{12}Z_Y Z_2 - r_{y2}Z_2 Z_1 + r_{y2}r_{12}Z_2^2] \\
&= E[Z_Y Z_1] - r_{12} E[Z_Y Z_2] - r_{y2}E[Z_2 Z_1] + r_{y2} r_{12} E{Z_2^2} \\  
&= r_{y1} - r_{12}r_{y2} - r_{y2}r_{12} + r_{y2}r_{12} \\
&= r_{y1} - r_{y2}r_{12}
\end{aligned}

Putting it all together and we convert back to unstandarized
\(X_1, X_2\) , we find

\begin{aligned}
\beta_1 &= \dfrac{r_{y1} - r_{y2} r_{12}}{1-r^2_{12}}
\end{aligned}

\begin{aligned}
\hat{\beta_1} &= \beta_1 \cdot \dfrac{S_Y}{S_{X_1}} \\
&= \beta_1 \cdot \sqrt{\dfrac{s_{Y}^2}{s_{X_1}^2}} \\
&= \dfrac{r_{y1} - r_{y2} r_{12}}{1-r^2_{12}} \cdot \sqrt{\dfrac{s_{Y}^2}{s_{X_1}^2}} \\
&= \dfrac{r_{y1}\dfrac{S_Y}{S_{X_1}} - \sqrt{\dfrac{s_{Y}^2}{s_{X_1}^2}} r_{y2} r_{12}}{1-r^2_{12}} \\
&= \dfrac{\hat{\beta_1'} - \sqrt{\dfrac{s_{Y}^2}{s_{X_1}^2}} r_{y2} r_{12}}{1-r^2_{12}}
\end{aligned}

where the model coefficient for the slop given Y regressed on only
\(X_1\) is \(\hat{\beta_1'} = r_{y1}\dfrac{S_Y}{S_{X_1}}\).

Therefore, we have computed the relationship of the parameters between
the model based only on \(X_1\) and the ordinary regression model based
on both \(X_1\) and \(X_2\).

\subsection{Problem 5}\label{problem-5}

The dataset rock in the faraway library contains 48 rock samples
obtained from twelve core samples from petroleum reservoirs sampled a
four cross sections. Each rock was measured for permeability.
Characteristics of the rock were its total perimeter of pores, total
area of pores, and shape. Your goal is to determine the ``best'' linear
model using these characteristics. Faraway (page 24) summarizes various
steps to consider in this model selection process. Please describe your
steps to derive the best model, what is your best model and any output
and figures to support this model.

\subsubsection{Begin with data
exploration}\label{begin-with-data-exploration}

\begin{Shaded}
\begin{Highlighting}[]
\CommentTok{\#Clear the workspace}
\FunctionTok{rm}\NormalTok{(}\AttributeTok{list=}\FunctionTok{ls}\NormalTok{())}
\FunctionTok{library}\NormalTok{(faraway)}
\FunctionTok{library}\NormalTok{(MASS)}

\FunctionTok{data}\NormalTok{(rock)}
\FunctionTok{summary}\NormalTok{(rock)}
\end{Highlighting}
\end{Shaded}

\begin{verbatim}
##       area            peri            shape              perm        
##  Min.   : 1016   Min.   : 308.6   Min.   :0.09033   Min.   :   6.30  
##  1st Qu.: 5305   1st Qu.:1414.9   1st Qu.:0.16226   1st Qu.:  76.45  
##  Median : 7487   Median :2536.2   Median :0.19886   Median : 130.50  
##  Mean   : 7188   Mean   :2682.2   Mean   :0.21811   Mean   : 415.45  
##  3rd Qu.: 8870   3rd Qu.:3989.5   3rd Qu.:0.26267   3rd Qu.: 777.50  
##  Max.   :12212   Max.   :4864.2   Max.   :0.46413   Max.   :1300.00
\end{verbatim}

\begin{Shaded}
\begin{Highlighting}[]
\FunctionTok{pairs}\NormalTok{(rock, }\AttributeTok{main=}\StringTok{"Scatterplot Matrix of Rock Data"}\NormalTok{)}
\end{Highlighting}
\end{Shaded}

\pandocbounded{\includegraphics[keepaspectratio]{STAT526_HW1_files/figure-latex/unnamed-chunk-3-1.pdf}}

\begin{Shaded}
\begin{Highlighting}[]
\CommentTok{\# Use Box{-}Cox to determine best transformation for perm}
\NormalTok{lmod }\OtherTok{\textless{}{-}} \FunctionTok{lm}\NormalTok{(perm }\SpecialCharTok{\textasciitilde{}}\NormalTok{ area }\SpecialCharTok{+}\NormalTok{ peri }\SpecialCharTok{+}\NormalTok{ shape, }\AttributeTok{data =}\NormalTok{ rock)}
\NormalTok{bc }\OtherTok{\textless{}{-}} \FunctionTok{boxcox}\NormalTok{(lmod, }\AttributeTok{plotit =} \ConstantTok{TRUE}\NormalTok{)}
\end{Highlighting}
\end{Shaded}

\pandocbounded{\includegraphics[keepaspectratio]{STAT526_HW1_files/figure-latex/unnamed-chunk-3-2.pdf}}

\begin{Shaded}
\begin{Highlighting}[]
\CommentTok{\# 3. Identify the exact lambda that maximizes the log{-}likelihood}
\NormalTok{lambda }\OtherTok{\textless{}{-}}\NormalTok{ bc}\SpecialCharTok{$}\NormalTok{x[}\FunctionTok{which.max}\NormalTok{(bc}\SpecialCharTok{$}\NormalTok{y)]}
\FunctionTok{print}\NormalTok{(lambda)}
\end{Highlighting}
\end{Shaded}

\begin{verbatim}
## [1] 0.2222222
\end{verbatim}

It seems like we should probably choose lambda in the confidence
interval. For interpretability, it is probably best to pick 0, which is
close to being in the 95\% confidence interval. We can do a quick
initial analysis to check this.

\begin{Shaded}
\begin{Highlighting}[]
\CommentTok{\# Compare Base Model vs. Log{-}Transformed Model}
\NormalTok{full\_model }\OtherTok{\textless{}{-}} \FunctionTok{lm}\NormalTok{(perm }\SpecialCharTok{\textasciitilde{}}\NormalTok{ area }\SpecialCharTok{+}\NormalTok{ peri }\SpecialCharTok{+}\NormalTok{ shape, }\AttributeTok{data =}\NormalTok{ rock)}
\NormalTok{log\_model  }\OtherTok{\textless{}{-}} \FunctionTok{lm}\NormalTok{(}\FunctionTok{log}\NormalTok{(perm) }\SpecialCharTok{\textasciitilde{}}\NormalTok{ area }\SpecialCharTok{+}\NormalTok{ peri }\SpecialCharTok{+}\NormalTok{ shape, }\AttributeTok{data =}\NormalTok{ rock)}

\FunctionTok{par}\NormalTok{(}\AttributeTok{mfrow=}\FunctionTok{c}\NormalTok{(}\DecValTok{1}\NormalTok{,}\DecValTok{2}\NormalTok{))}
\FunctionTok{plot}\NormalTok{(full\_model, }\AttributeTok{which=}\DecValTok{1}\NormalTok{, }\AttributeTok{main=}\StringTok{"Original Scale"}\NormalTok{)}
\FunctionTok{plot}\NormalTok{(log\_model, }\AttributeTok{which=}\DecValTok{1}\NormalTok{, }\AttributeTok{main=}\StringTok{"Log Scale"}\NormalTok{)}
\end{Highlighting}
\end{Shaded}

\pandocbounded{\includegraphics[keepaspectratio]{STAT526_HW1_files/figure-latex/unnamed-chunk-4-1.pdf}}

Now perhaps a step-wise AIC model selection:

\begin{Shaded}
\begin{Highlighting}[]
\NormalTok{best\_mod }\OtherTok{\textless{}{-}} \FunctionTok{step}\NormalTok{(log\_model, }\AttributeTok{direction =} \StringTok{"both"}\NormalTok{)}
\end{Highlighting}
\end{Shaded}

\begin{verbatim}
## Start:  AIC=-11.54
## log(perm) ~ area + peri + shape
## 
##         Df Sum of Sq    RSS     AIC
## - shape  1     0.727 32.675 -12.460
## <none>               31.949 -11.539
## - area   1    22.788 54.736  12.304
## - peri   1    53.988 85.937  33.956
## 
## Step:  AIC=-12.46
## log(perm) ~ area + peri
## 
##         Df Sum of Sq     RSS     AIC
## <none>                32.675 -12.460
## + shape  1     0.727  31.949 -11.539
## - area   1    28.977  61.652  16.015
## - peri   1    81.412 114.088  45.557
\end{verbatim}

\begin{Shaded}
\begin{Highlighting}[]
\FunctionTok{summary}\NormalTok{(best\_mod)}
\end{Highlighting}
\end{Shaded}

\begin{verbatim}
## 
## Call:
## lm(formula = log(perm) ~ area + peri, data = rock)
## 
## Residuals:
##     Min      1Q  Median      3Q     Max 
## -1.9801 -0.5936  0.1406  0.6637  1.4581 
## 
## Coefficients:
##               Estimate Std. Error t value Pr(>|t|)    
## (Intercept)  5.746e+00  3.621e-01  15.867  < 2e-16 ***
## area         5.144e-04  8.143e-05   6.317 1.05e-07 ***
## peri        -1.616e-03  1.526e-04 -10.589 8.41e-14 ***
## ---
## Signif. codes:  0 '***' 0.001 '**' 0.01 '*' 0.05 '.' 0.1 ' ' 1
## 
## Residual standard error: 0.8521 on 45 degrees of freedom
## Multiple R-squared:  0.7426, Adjusted R-squared:  0.7311 
## F-statistic:  64.9 on 2 and 45 DF,  p-value: 5.49e-14
\end{verbatim}

Our best model keeps area and perimeter. We can now check the
diagnostics.

\begin{Shaded}
\begin{Highlighting}[]
\FunctionTok{par}\NormalTok{(}\AttributeTok{mfrow=}\FunctionTok{c}\NormalTok{(}\DecValTok{2}\NormalTok{,}\DecValTok{2}\NormalTok{))}
\FunctionTok{plot}\NormalTok{(best\_mod)}
\end{Highlighting}
\end{Shaded}

\pandocbounded{\includegraphics[keepaspectratio]{STAT526_HW1_files/figure-latex/unnamed-chunk-6-1.pdf}}

Looking at the plots, we try to ascertain whether our assumptions for
the model hold true. In this case we see no distinct pattern in the
residuals vs fitted plot and the Normal QQ plot shows a diagonal line,
showing normal distribution. Things look good.

We can interpret our final model. Positive coefficient for area means
that larger pores generally increase permeability and negative
coefficient for perimeter means that for a fixed area, a larger
perimeter implies lower permeability.

\end{document}
